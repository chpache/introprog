%!TEX encoding = UTF-8 Unicode
\documentclass[a4paper]{compendium}
%\usepackage{xr} %to crossreference to compendium1.tex
\externaldocument{compendium1}
\usepackage[french]{babel}


\addto\captionsswedish{%
  \renewcommand{\appendixname}{Appendix}%
}
%TODO: Glossary
%http://tex.stackexchange.com/questions/5821/creating-a-standalone-glossary/5837#5837

\setlength{\columnsep}{16mm}

\input{global-constants.tex}

\title{
{\vspace{-3.0cm}\bf\sffamily\Huge\selectfont  Introduction à la programmation avec Scala}
\\ \vspace{2em}%\hspace*{1.5cm}\inputgraphics[width=0.6\textwidth]{../img/gurka} \\
{\sffamily \textbf{Kompendium 2}\\Andra läsperioden: Modul 8 -- 14}\\\vspace{2cm}
\includegraphics[height=11cm]{../img/glider-blinker-block}
%\includegraphics[height=4cm]{../img/scala-logo.png}
%\includegraphics[height=4cm]{../img/java-logo.png}
%\includegraphics[height=12cm]{cover/gurka.jpg}
}

\author{Björn Regnell}
\date{\raggedbottom%
\vspace{1em}\begin{minipage}{1.0\textwidth}\centering
EDAA45, Lp1-2, HT \CurrentYear\\
Datavetenskap, LTH\\
Lunds universitet\\
~\\
Kompileringsdatum: \today \\
\url{http://cs.lth.se/pgk}
\end{minipage}
}

\usepackage{multicol}

\usepackage{pgffor}  %% http://stackoverflow.com/questions/2561791/iteration-in-latex
%  allows:  \foreach \n in {1,...,4}{ do something with \n }

\usepackage{framed}  %  allows:   \begin{framed}\end{framed}
\FrameSep5pt
\OuterFrameSep0pt

% \newenvironment{Slide}[2][]{%
% \begin{oframed}\setlist{noitemsep}%
% {\vspace{-1.5\topsep}}%tighter frames
% \subsection{#2}%
% }%
% {\end{oframed}} 

\newenvironment{Slide}[2][]{%
%\noindent\rule{\textwidth}{0.4pt}%
\setlist{noitemsep}%
%{\vspace{-1.5\topsep}}%tighter frames
\subsection{#2}%
}%
{~\newline\noindent\rule{\textwidth}{0.4pt}}

% \newcommand{\SlideHeading}[1]{\section*{#1}}

% \usepackage[most]{tcolorbox}
% \newenvironment{Slide}[2][]
%   {\vspace{0.5em}\begin{tcolorbox}[left=1.5em,%width=1.05\textwidth,
%   grow to right by=0.05\textwidth,grow to left by=0.05\textwidth,%
%   %breakable,
%   %frame hidden,
%   colframe=gray!20,
%   enhanced]\setlist{noitemsep}\SlideHeading{#2}}
%   {\end{tcolorbox}\vspace{0.5em}}

\newcommand{\Subsection}[1]{} %ignore slide sections
\newcommand{\SlideOnly}[1]{} %ignore slide font size

\usepackage[framemethod=tikz]{mdframed}


\newif\ifkompendium  % to allow conditional text in slides only showing up in compendium
\kompendiumtrue      % in slides: \kompendiumfalse

\newif\ifPreSolution  % to allow tasks and solutions in same file
\PreSolutiontrue      % in solutions: \PreSolutionfalse

\let\QUESTBEGIN\ifPreSolution  % to mark formatting and numbering of exercises
\let\SOLUTION\else  % to mark solutions in the same file as questions
\let\QUESTEND\fi    % to mark end of exercise

\input{generated/names-generated.tex}

\begin{document}

\pagenumbering{roman}

\frontmatter
\maketitle
\input{prechapters/licence-contributors.tex}
\input{prechapters/progress-forms.tex}
\input{prechapters/preface-compendium2.tex}

\setcounter{tocdepth}{2} % set headings level in table of contents
\tableofcontents
\mainmatter

\pagenumbering{arabic}


\part{Modulöversikt}

\begin{table}
\noindent\resizebox{1.0\columnwidth}{!}{
\renewcommand{\arraystretch}{2.0}
%!TEX encoding = UTF-8 Unicode
\begin{tabular}{l|l|l|l}
\textit{W} & \textit{Modul} & \textit{Övn} & \textit{Lab} \\ \hline \hline
W01 & Introduction & expressions & kojo \\
W02 & Program & programs & -- \\
W03 & Funktioner & functions & irritext \\
W04 & Objekt & objects & blockmole \\
W05 & Klasser & classes & -- \\
W06 & Mönster, undantag & patterns & blockbattle \\
W07 & Sekvenser & sequences & shuffle \\
KS & KONTROLLSKRIVN. & -- & -- \\
W08 & Matriser, typparametrar & matrices & life \\
W09 & Mängder, tabeller & lookup & words \\
W10 & Arv & inheritance & snake \\
W11 & Språkskillnader & scala-java & javatext \\
W12 & Sortering & sort & -- \\
W13 & Repetition, tentaträning, projekt & examprep & Projekt \\
W14 & Extra & extra & -- \\
T & TENTAMEN & -- & -- \\
\end{tabular}

}
\end{table}
\clearpage

\hyphenation{intro-duktion sekvens-algoritmer kod-strukturer data-strukturer}
{\fontsize{11}{12}\selectfont
\renewcommand{\arraystretch}{1.75}
\begin{longtable}{@{}p{.05\textwidth} | >{\hspace{0.1em}\raggedright\bfseries\sffamily}p{.15\textwidth}  >{\raggedleft\arraybackslash\hspace{0.0em}%\fontsize{10.5}{12}\selectfont
}p{0.735\textwidth}}
W01 & Introduction & sekvens, alternativ, repetition, abstraktion, editera, kompilera, exekvera, datorns delar, virtuell maskin, litteral, värde, uttryck, identifierare, variabel, typ, tilldelning, namn, val, var, def, definera och anropa funktion, funktionshuvud, funktionskropp, procedur, inbyggda grundtyper, Int, Long, Short, Double, Float, Byte, Char, String, println, typen Unit, enhetsvärdet (), stränginterpolatorn s, if, else, true, false, MinValue, MaxValue, aritmetik, slumptal, math.random, logiska uttryck, de Morgans lagar, while-sats, for-sats \\
W02 & Program & kompilerad app, skript, main i Scala, scalac, utdata, println, indata, scala.io.StdIn.readLine, programargument, args i main, main i Java, javac, java.lang.System.out.println, iterera över element i samling, for-uttryck, yield, map, foreach, samling, sekvens, indexering, Array, Vector, intervall, Range, algoritm vs implementation, pseudokod, algoritm: SWAP, algoritm: SUM, algoritm: MIN/MAX, algoritm: MININDEX \\
W03 & Funktioner & parameter, argument, returtyp, default-argument, namngivna argument, parameterlista, funktionshuvud, funktionskropp, applicera funktion på alla element i en samling, uppdelad parameterlista, skapa egen kontrollstruktur, funktionsvärde, funktionstyp, äkta funktion, stegad funktion, apply, anonyma funktioner, lambda, aktiveringspost, anropsstacken, objektheapen, funktioner är objekt med apply-metod, rekursion, scala.util.Random, slumptalsfrö \\
W04 & Objekt & modul, singelobjekt, paket, punktnotation, tillstånd, medlem, attribut, metod, paket, import, filstruktur, jar, dokumentation, programlayout, JDK, import, selektiv import, namnbyte vid import, tupel, multipla returvärden, block, lokal variabel, skuggning, lokal funktion, namnrymd, synlighet, privat medlem, inkapsling, getter och setter, principen om uniform access, överlagring av metoder, introprog.PixelWindow, initialisering, lazy val, värdeandrop, namnanrop, typalias \\
W05 & Klasser & objektorientering, klass, instans, Point, Square, Complex, Any, isInstanceOf, toString, new, null, this, accessregler, private, private[this], klassparameter, primär konstruktor, fabriksmetod, alternativ konstruktor, förändringsbar, oföränderlig, case-klass, kompanjonsobjekt, referenslikhet, innehållslikhet, eq, == \\
W06 & Mönster, undantag & mönstermatchning, match, Option, throw, try, catch, Try, unapply, sealed, flatten, flatMap, partiella funktioner, collect, wildcard-mönster, variabelbindning i mönster, sekvens-wildcard, bokstavliga mönster, implementera equals, hashcode, exempel: equals för klassen Complex, switch-sats i Java \\
W07 & Sekvenser & översikt av Scalas samlingsbibliotek och samlingsmetoder, klasshierarkin i scala.collection, Traversable, Iterable, Seq, List, ListBuffer, ArrayBuffer, WrappedArray, sekvensalgoritm, algoritm: SEQ-COPY, in-place vs copy, algoritm: SEQ-REVERSE, registrering, algoritm: SEQ-REGISTER, linjärsökning, algoritm: LINEAR-SEARCH, tidskomplexitet, minneskomplexitet, sekvenser i Java vs Scala, for-sats i Java, java.util.Scanner, översikt strängmetoder, StringBuilder, ordning, inbyggda sökmetoder, find, indexOf, indexWhere, inbyggda sorteringsmetoder, sorted, sortWith, sortBy, repeterade parametrar \\
KS & \multicolumn{2}{l}{KONTROLLSKRIVN.} \\
W08 & Matriser, typparametrar & matris, nästlad samling, nästlad for-sats, typparameter, generisk funktion, generisk klass, fri vs bunden typparameter, generisk samling som typparameter, matriser i Java vs Scala, allokering av nästlade arrayer i Scala och Java \\
W09 & Mängder, tabeller & innehållstest, mängd, Set, mutable.Set, nyckel-värde-tabell, Map, mutable.Map, hash code, java.util.HashMap, java.util.HashSet, persistens, serialisering, textfiler, Source.fromFile, java.nio.file, repetition inför kontrollskrivning \\
W10 & Arv & arv, polymorfism, trait, extends, asInstanceOf, with, inmixning, supertyp, subtyp, bastyp, override, Scalas typhierarki, Any, AnyRef, Object, AnyVal, Null, Nothing, topptyp, bottentyp, referenstyper, värdetyper, Shape som bastyp till Rectangle och Circle, accessregler vid arv, protected, final, case-object, typer med uppräknade värden, trait, abstrakt klass, inmixning, gränssnitt, interface i Java, programmeringsgränssnitt (api) \\
W11 & Språkskillnader & syntaxskillnader mellan Scala och Java, klasser i Scala och Java, referensvariabler i Java, enkla värden i Java, primitiva typer i Java, referenstilldelning och värdetilldelning i Java, alternativ konstruktor i Scala och Java, for-sats i Java, for-each-sats i Java, java.util.ArrayList, autoboxing i Java, wrapperklasser i Java, samlingar i Java, scala.collection.JavaConverters, namnkonventioner för konstanter i Scala och Java, kodläsbarhet, idiom, kodningsstandard \\
W12 & Sortering & strängjämförelse, compareTo, implicit ordning, binärsökning, algoritm: BINARY-SEARCH, sortering till ny vektor, sortering på plats, insättningssortering, urvalssortering, algoritm: INSERTION-SORT, algoritm: SELECTION-SORT, Ordering[T], Ordered[T], Comparator[T], Comparable[T], riktlinjer för projektredovisning \\
W13 & Repetition, tentaträning, projekt & göra extenta, förbereda projektredovisning, skapa dokumentation med scaladoc och javadoc \\
W14 & Extra & tråd, jämlöpande exekvering, icke-blockerande anrop, callback, java.lang.Thread, java.util.concurrent.atomic.AtomicInteger, scala.concurrent.Future, kort om html+css+javascript+scala.js och webbprogrammering \\
T & \multicolumn{2}{l}{TENTAMEN} \\
\end{longtable}
}

%\renewcommand{\SlideHeading}[1]{\subsection{#1}}  %numbering sections in compendium slides

\part{Moduler}

\setcounter{chapter}{7}

\input{modules/w08-matrices-chapter.tex}
%\input{generated/w08-chaphead-generated.tex}
\input{modules/w08-matrices-exercise.tex}
\input{modules/w08-matrices-lab.tex}

\input{modules/w09-setmap-chapter.tex}
%\input{generated/w09-chaphead-generated.tex}
\input{modules/w09-setmap-exercise.tex}
\input{modules/w09-setmap-lab.tex}

\input{modules/w10-inheritance-chapter.tex}
%\input{generated/w10-chaphead-generated.tex}
\input{modules/w10-inheritance-exercise.tex}
\input{modules/w10-inheritance-lab.tex}

\input{modules/w11-scalajava-chapter.tex}
%\input{generated/w11-chaphead-generated.tex}
\input{modules/w11-scalajava-exercise.tex}
\input{modules/w11-scalajava-lab.tex}

\input{modules/w12-sorting-chapter.tex}
%\input{generated/w12-chaphead-generated.tex}
\input{modules/w12-sorting-exercise.tex}
\input{modules/w12-sorting-lab.tex}

\input{modules/w13-examprep-chapter.tex}
%\input{generated/w13-chaphead-generated.tex}
\input{modules/w13-examprep-exercise.tex}
\input{modules/w13-assignment-bank.tex}
\input{modules/w13-assignment-tabular.tex}
\input{modules/w13-assignment-music.tex}
\input{modules/w13-assignment-imageprocessing.tex}

\input{modules/w14-extra-chapter.tex}
%\input{generated/w14-chaphead-generated.tex}
\input{modules/w14-extra-exercise.tex}
\input{modules/w14-extra-lab.tex}


\part{Appendix}
\appendix

%\setcounter{chapter}{8} %next after 8 is I in \Alph
%\setcounter{chapter}{3} %next after 3 is D in \Alph
\input{postchapters/kojo.tex}
\input{postchapters/terminal.tex}
\input{postchapters/compile.tex}
\input{postchapters/debug.tex}
\input{postchapters/document.tex}
\input{postchapters/build.tex}
\input{postchapters/version-control.tex}
\input{postchapters/ide.tex}
%\input{postchapters/scalajs.tex} %TODO!!
%\input{postchapters/android.tex} %TODO!!
%\input{postchapters/vbox.tex}  %TODO!!

\part{Lösningar}

\setcounter{chapter}{11} %next is L in \Alph
\chapter{Lösningar till övningarna}\label{chapter:solutions}
\setcounter{section}{7}

\PreSolutionfalse

\let\QUESTBEGIN\ifPreSolution  % to mark formatting and numbering of exercises
\let\SOLUTION\else  % to mark solutions in the same file as questions
\let\QUESTEND\fi    % to mark end of exercise

\input{modules/w08-matrices-exercise.tex}
\input{modules/w09-setmap-exercise.tex}
\input{modules/w10-inheritance-exercise.tex}
\input{modules/w11-scalajava-exercise.tex}
\input{modules/w12-sorting-exercise.tex}
\input{modules/w13-examprep-exercise.tex}
\input{modules/w14-extra-exercise.tex}


%\chapter{Snabbreferens}\label{chapter:quickref}
%
%Detta appendix innehåller en snabbreferens för Scala och Java. Snabbreferensen är enda tillåtna hjälpmedel under kursens skriftliga tentamen.
%
%Lär dig vad som finns i snabbreferensen så att du snabbt hittar det du behöver och träna på hur du  effektivt kan dra nytta av den när du skriver program med papper och penna utan datorhjälpmedel.
%
%\clearpage
%~
%\clearpage
%
%\includepdf[pages={1-12}, scale=0.77, frame]{../quickref/quickref.pdf}


\end{document}
