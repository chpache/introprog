%!TEX encoding = UTF-8 Unicode
\documentclass[a4paper]{compendium}
\usepackage[french]{babel}

%\usepackage{xr} %to crossreference ???
\externaldocument{compendium} %to crossreference to compendium.tex

\addto\captionsswedish{%
  \renewcommand{\appendixname}{Appendix}%
}
%TODO: Glossary
%http://tex.stackexchange.com/questions/5821/creating-a-standalone-glossary/5837#5837

\setlength{\columnsep}{16mm}

\input{global-constants.tex}

\title{
{\vspace{-3.0cm}\bf\sffamily\Huge\selectfont  Introduktion till programmering med Scala}
\\ \vspace{1em}%\hspace*{1.5cm}\inputgraphics[width=0.6\textwidth]{../img/gurka} \\
{\sffamily  Föreläsningar}\\\vspace{2cm}
%\includegraphics[height=4cm]{../img/scala-logo.png}
%\includegraphics[height=4cm]{../img/java-logo.png}
\includegraphics[height=12cm]{cover/gurka.jpg}
}

%\author{Redaktör: Björn Regnell}
\date{\raggedbottom%
\vspace{-2em}\begin{minipage}{1.0\textwidth}\centering
EDAA45, Lp1-2, HT \CurrentYear\\
Datavetenskap, LTH\\
Lunds Universitet\\
~\\
Kompileringsdatum: \today \\
\url{http://cs.lth.se/pgk}
\end{minipage}
}

\usepackage{multicol}

\usepackage{tikz}
\usetikzlibrary{shapes.geometric, shapes.symbols, arrows, matrix, shapes, positioning, calc}
\usepackage{tkz-euclide}\usetkzobj{all}%https://tex.stackexchange.com/questions/96459/automatically-draw-and-labels-angles-of-a-triangle-in-tikz

\usepackage{pgffor}  %% http://stackoverflow.com/questions/2561791/iteration-in-latex
                     %  allows:  \foreach \n in {1,...,4}{ do something with \n }

\usepackage{framed}  %  allows:   \begin{framed}\end{framed}
%\newenvironment{Slide}[2][]
%  {\begin{framed}\setlist{noitemsep}\section*{#2}}
%  {\end{framed}}


\newcommand{\SlideHeading}[1]{\section*{#1}}

\usepackage[most]{tcolorbox}
\newenvironment{Slide}[2][]
  {\vspace{0.5em}\begin{tcolorbox}[left=1.5em,%width=1.05\textwidth,
  grow to right by=0.05\textwidth,grow to left by=0.05\textwidth,%
  %breakable,
  %frame hidden,
  colframe=gray!20,
  enhanced]\setlist{noitemsep}\SlideHeading{#2}}
  {\end{tcolorbox}\vspace{0.5em}}

\newcommand{\Subsection}[1]{} %ignore slide sections
\newcommand{\SlideOnly}[1]{} %ignore slide font size

\usepackage[framemethod=tikz]{mdframed}

\newif\ifkompendium  % to allow conditional text in slides only showing up in compendium
\kompendiumtrue      % in slides: \kompendiumfalse

\newif\ifPreSolution  % to allow tasks and solutions in same file
\PreSolutiontrue      % in solutions: \PreSolutionfalse

\input{generated/names-generated.tex}

\begin{document}

\pagenumbering{roman}

\frontmatter
\maketitle
\input{prechapters/licence-contributors.tex}
\input{prechapters/progress-forms.tex}
\input{prechapters/preface.tex}

\setcounter{tocdepth}{2} % set headings level in table of contents
\tableofcontents
\mainmatter

\pagenumbering{arabic}

\part{Om kursen}
\setcounter{chapter}{-3}
\input{prechapters/course-architecture.tex}
\input{prechapters/course-instructions.tex}
\input{prechapters/how-to-contribute.tex}

%\renewcommand{\SlideHeading}[1]{\subsection{#1}}  %numbering sections in compendium slides

\part{Moduler}

\input{modules/w01-intro-chapter.tex}
\input{modules/w02-programs-chapter.tex}
\input{modules/w03-functions-chapter.tex}
\input{modules/w04-objects-chapter.tex}
\input{modules/w05-classes-chapter.tex}
\input{modules/w06-patterns-chapter.tex}
\input{modules/w07-sequences-chapter.tex}

\input{modules/w08-matrices-chapter.tex}
\input{modules/w09-setmap-chapter.tex}
\input{modules/w10-inheritance-chapter.tex}
\input{modules/w11-scalajava-chapter.tex}
\input{modules/w12-sorting-chapter.tex}
\input{modules/w13-examprep-chapter.tex}
\input{modules/w14-extra-chapter.tex}

\part{Appendix}
\appendix
\input{postchapters/kojo.tex}
\input{postchapters/terminal.tex}
\input{postchapters/compile.tex}
\input{postchapters/debug.tex}
\input{postchapters/document.tex}
\input{postchapters/build.tex}
\input{postchapters/version-control.tex}
\input{postchapters/vbox.tex}
\input{postchapters/ide.tex}
\input{postchapters/scalajs.tex}
\input{postchapters/android.tex}

%\chapter{Ordlista}

%\chapter{Lösningar till övningarna}\label{chapter:solutions}
%\foreach \n in {1,...,9}{%
%  \input{modules/w0\n-solutions.tex}
%}
%\foreach \n in {10,...,14}{%
%  \input{modules/w\n-solutions.tex}
%}
%
%\chapter{Snabbreferens}\label{chapter:quickref}
%
%Detta appendix innehåller en snabbreferens för Scala och Java. Snabbreferensen är enda tillåtna hjälpmedel under kursens skriftliga tentamen.
%
%Lär dig vad som finns i snabbreferensen så att du snabbt hittar det du behöver och träna på hur du  effektivt kan dra nytta av den när du skriver program med papper och penna utan datorhjälpmedel.
%
%\clearpage
%~
%\clearpage
%
%\includepdf[pages={1-12}, scale=0.77, frame]{../quickref/quickref.pdf}


\end{document}
