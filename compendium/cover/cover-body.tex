
\pagecolor{Background}
\color{Foreground}


\newcommand{\LeftMarginBack}{1.0}
\newcommand{\RightMarginFront}{1.0}
\newcommand{\YPosFront}{-1.5}
\newcommand{\YPosBack}{0}

\thispagestyle{empty}
\renewcommand{\familydefault}{\sfdefault}
\renewcommand{\arraystretch}{1.5}
%\newgeometry{a3paper,landscape, centering}
%\makeatletter\CROP@center\makeatother

\begin{tikzpicture}[overlay,remember picture]%[overlay,remember picture]
    %\fill[Pink,fill opacity=1.0] (current page.south west) rectangle (current page.north east);
    \node[text width=250mm, align=right] (intro) at ($(current page.north east)+(-14.5,-6.8)-(\RightMarginFront,\YPosFront)$) {
       \fontsize{28}{46}\sffamily\selectfont{Introduction~~till}
       };

    \node[text width=250mm, align=right] (prog) at ($(current page.north east)+(-14.5,-8.5)-(\RightMarginFront,\YPosFront)$) {
       \fontsize{52}{46}\sffamily\selectfont\textbf{programmering}
       };

       \node[text width=250mm, align=right] (intro) at ($(current page.north east)+(-14.5,-10.0)-(\RightMarginFront,\YPosFront)$) {
          \fontsize{28}{46}\sffamily\selectfont{med~~Scala}
          };

          \node[text width=250mm, align=right] (prog) at ($(current page.north east)+(-14.5,-12.5)-(\RightMarginFront,\YPosFront)$) {
            \fontsize{18}{40}\sffamily\selectfont\textit{2019\hspace{0.7em}{\CurrentPart}} 
            };
     


          \node[text width=250mm, align=right] (intro) at ($(current page.north east)+(-14.5,-28.0)-(\RightMarginFront,\YPosFront)$) {
            \fontsize{24}{46}\sffamily\selectfont{Björn Regnell}
            };
  

    \node[text width=250mm, align=right] (prog) at ($(current page.north east)+(-14.5,-30.0)-(\RightMarginFront,\YPosFront)$) {
       \fontsize{18}{46}\sffamily\selectfont\texttt{http://cs.lth.se/pgk}
       };

    %\node[text width=250mm, align=right] (prog) at ($(current page.north east)+(-14.5,-28.2)$) {
    %   \fontsize{18}{46}\sffamily\selectfont  2016
    %   };


    %\node[above right] (picture) at ($(current page.north east)+(-16,-27.5)-(\RightMarginFront,\YPosFront)$)
     %{\includegraphics[width=140mm]{gurka.jpg}};
    %\node[above right] (picture) at ($(current page.north east)+(-15,-25.5)-(\RightMarginFront,\YPosFront)$){\includegraphics[width=120mm]{../../img/scala-logo}};
    \node[above right] (picture) at ($(current page.north east)+(-14.15,-25.5)-(\RightMarginFront,\YPosFront)$){\CoverPicture};


\node[above right] (logo) at ($(current page.west)+(19.0,-21.8)+(2.5,0.5)$) {\includegraphics[scale=0.54]{../../img/logoLUeng}};



\node[anchor=north,rotate=-90, align=left] (back) at ($(current page.north)+(0.0,-14)+(0,+2.0)$) {
\fontsize{16}{20}\sffamily\selectfont\textbf{Introduction à la programmation avec Scala%
 %och Java
 }\hspace{2em}\textit{Björn Regnell}
 };

\node[anchor=north] (back) at ($(current page.north)+(-0.4,-16.0)+(0,-6.3)$) {
\fontsize{13}{20}\sffamily\selectfont {\CurrentYear}};

\node[anchor=north] (back) at ($(current page.north)+(-0.4,-16.0)+(0,-7.3)$) {
\fontsize{13}{20}\sffamily\selectfont {\CurrentPart}};


\node[above right, scale=1.3] (modulplan) at ($(current page.west)+(\LeftMarginBack,-11.5)+(1.0,\YPosBack)$) {
\sffamily%!TEX encoding = UTF-8 Unicode
\begin{tabular}{l|l|l|l}
\textit{W} & \textit{Modul} & \textit{Övn} & \textit{Lab} \\ \hline \hline
W01 & Introduction & expressions & kojo \\
W02 & Program & programs & -- \\
W03 & Funktioner & functions & irritext \\
W04 & Objekt & objects & blockmole \\
W05 & Klasser & classes & -- \\
W06 & Mönster, undantag & patterns & blockbattle \\
W07 & Sekvenser & sequences & shuffle \\
KS & KONTROLLSKRIVN. & -- & -- \\
W08 & Matriser, typparametrar & matrices & life \\
W09 & Mängder, tabeller & lookup & words \\
W10 & Arv & inheritance & snake \\
W11 & Språkskillnader & scala-java & javatext \\
W12 & Sortering & sort & -- \\
W13 & Repetition, tentaträning, projekt & examprep & Projekt \\
W14 & Extra & extra & -- \\
T & TENTAMEN & -- & -- \\
\end{tabular}

};

\node[above right, text width=15cm,align=left] (collection-traits) at ($(current page.west)+(\LeftMarginBack,3.8)+(1.0,\YPosBack)$) {
\begin{minipage}{1.0\textwidth}\sffamily\large
Detta kompendium är kurslitteratur i grundkursen i programmering på Datateknikprogrammet vid Lunds tekniska högskola.

\vspace{1em}
I kursen används det moderna och kraftfulla programmeringsspråket Scala för att illustrera grunderna i imperativ och objektorienterad programmering, samt elementär funktions-programmering.

\vspace{1em}
En viktig framgångsfaktor vid studier i programmering är din egen upptäckarglädje och experimentlusta. Du lär dig bäst genom att skapa dina egna program. Varje bugg du fixar i din egen kod fördjupar dina insikter.

\vspace{1em}
Läromaterialet fokuserar därför på lärande genom praktiskt programmeringsarbete och innehåller övningar och laborationer som är organiserade i moduler enligt en noga uttänkt progression, där dina kunskaper utvecklas steg för steg. Varje modul har ett tema enligt tabellen nedan.



\vspace{1em}
Välkommen till datavetenskapens fascinerande värld!
\end{minipage}
};

\node[above right, scale=1.3] (modulplan) at ($(current page.west)+(\LeftMarginBack,-13.5)+(1.0,\YPosBack)$) {
\begin{minipage}{0.33\textwidth}
\sffamily Datavetenskap, LTH, Lunds universitet. Licens: CC-BY-SA. Upplaga \CurrentYear. \\ \texttt{http://cs.lth.se/pgk}
\end{minipage}
};


%\node[above right] (collection-traits) at ($(current page.west)+(4.0,-9.5)$) {\includegraphics[scale=0.55]{../../img/collection/collection-traits}};

%
%
%\node[above right] (collection-traits) at ($(current page.west)+(1.0,7.5)$) {\includegraphics[scale=0.55]{../../img/collection/collection-traits}};
%
%\node[above right] (collection-traits-text) at ($(current page.west)+(1.0,12.5)$) {\texttt{\large scala.collection}};
%
%\node[above right] (collection-legend) at ($(current page.west)+(14.0,7.5)$) {\includegraphics[scale=0.30]{../../img/collection/collection-legend}};
%
%\node[above right] (collection-immutable) at ($(current page.west)+(2,-3.4)$) {\includegraphics[scale=0.50]{../../img/collection/collection-immutable}};
%
%\node[above right] (collection-traits-text) at ($(current page.west)+(1.0,4.5)$) {\large\texttt{\large immutable}};
%
%\node[above right] (collection-mutable) at ($(current page.west)+(0,-15.2)$) {\includegraphics[scale=0.40]{../../img/collection/collection-mutable}};
%
%\node[above right] (collection-traits) at ($(current page.west)+(1.0,-5.5)$) {\texttt{\large mutable}};



% ../../img/collection/collection-immutable
\end{tikzpicture}