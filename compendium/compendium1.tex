%!TEX encoding = UTF-8 Unicode
\documentclass[a4paper]{compendium}

%\usepackage{xr} %to crossreference ???
\externaldocument{compendium2} %to crossreference to compendium2.tex

\usepackage[french]{babel}
\addto\captionsswedish{%
  \renewcommand{\appendixname}{Appendix}%
}
%TODO: Glossary
%http://tex.stackexchange.com/questions/5821/creating-a-standalone-glossary/5837#5837

\setlength{\columnsep}{16mm}

\input{global-constants.tex}

\title{
{\vspace{-3.0cm}\bf\sffamily\Huge\selectfont  Introduction à la programmation avec Scala}
\\ \vspace{2em}%\hspace*{1.5cm}\inputgraphics[width=0.6\textwidth]{../img/gurka} \\
{\sffamily \textbf{Kompendium 1}\\ Första läsperioden: Modul 1 -- 7}\\\vspace{3cm}
%\includegraphics[height=8cm]{../img/logoLUeng.pdf}
%\includegraphics[height=5cm]{../img/scala-icon.png}
\includegraphics[height=8cm]{../img/blockworm}
%\includegraphics[height=4cm]{../img/java-logo.png}
%\includegraphics[height=12cm]{cover/gurka.jpg}
\vspace{2cm}
}

\author{Björn Regnell}
\date{\raggedbottom%
\vspace{1em}\begin{minipage}{1.0\textwidth}\centering
EDAA45, Lp1-2, HT \CurrentYear\\
Datavetenskap, LTH\\
Lunds universitet\\
~\\
Kompileringsdatum: \today \\
\url{http://cs.lth.se/pgk}
\end{minipage}
}

\usepackage{multicol}

\usepackage{pgffor}  %% http://stackoverflow.com/questions/2561791/iteration-in-latex
                     %  allows:  \foreach \n in {1,...,4}{ do something with \n }

\usepackage{framed}  %  allows:   \begin{framed}\end{framed}
\FrameSep5pt
\OuterFrameSep0pt

% \newenvironment{Slide}[2][]{%
% \begin{oframed}\setlist{noitemsep}%
% {\vspace{-1.5\topsep}}%tighter frames
% \subsection{#2}%
% }%
% {\end{oframed}}

\newenvironment{Slide}[2][]{%
%\noindent\rule{\textwidth}{0.4pt}%
\setlist{noitemsep}%
%{\vspace{-1.5\topsep}}%tighter frames
\subsection{#2}%
}%
{~\newline\noindent\rule{\textwidth}{0.4pt}}

%\newenvironment{Slide}[2][]{\setlist{noitemsep}\subsection{#2}}{}


%\newcommand{\SlideHeading}[1]{\section*{#1}}

%\usepackage[most]{tcolorbox}
% \newenvironment{Slide}[2][]
%   {\vspace{0.5em}\begin{tcolorbox}[left=1.5em,%width=1.05\textwidth,
%   grow to right by=0.05\textwidth,grow to left by=0.05\textwidth,%
%   %breakable,
%   %frame hidden,
%   colframe=gray!20,
%   enhanced]\setlist{noitemsep}\SlideHeading{#2}}
%   {\end{tcolorbox}\vspace{0.5em}}

\newcommand{\Subsection}[1]{} %ignore slide sections
\newcommand{\SlideOnly}[1]{} %ignore slide font size

\usepackage[framemethod=tikz]{mdframed}

\newif\ifkompendium  % to allow conditional text in slides only showing up in compendium
\kompendiumtrue      % in slides: \kompendiumfalse

\newif\ifPreSolution  % to allow tasks and solutions in same file
\PreSolutiontrue      % in solutions: \PreSolutionfalse

\let\QUESTBEGIN\ifPreSolution  % to mark formatting and numbering of exercises
\let\SOLUTION\else  % to mark solutions in the same file as questions
\let\QUESTEND\fi    % to mark end of exercise


\input{generated/names-generated.tex}

\begin{document}

\pagenumbering{roman}

\frontmatter
\maketitle
\input{prechapters/licence-contributors.tex}
\input{prechapters/progress-forms.tex}
\input{prechapters/preface.tex}

\setcounter{tocdepth}{2} % set headings level in table of contents

\tableofcontents
\mainmatter

\pagenumbering{arabic}


\part{Om kursen}
\setcounter{chapter}{-3}
\input{prechapters/course-architecture.tex}
\input{prechapters/course-instructions.tex}
\input{prechapters/how-to-contribute.tex}

%\renewcommand{\SlideHeading}[1]{\subsection{#1}}  %numbering sections in compendium slides

\part{Moduler}

  %remove space before subsection
  %https://stackoverflow.com/questions/3191640/heading-subsection-at-start-of-framed-environment-in-latex-without-leading-padd


\input{modules/w01-intro-chapter.tex}
%%!TEX encoding = UTF-8 Unicode
\chapter{Introduction}\label{chapter:W01}
Begrepp som ingår i denna veckas studier:
\begin{multicols}{2}\begin{itemize}[noitemsep,label={$\square$},leftmargin=*]
\item sekvens
\item alternativ
\item repetition
\item abstraktion
\item editera
\item kompilera
\item exekvera
\item datorns delar
\item virtuell maskin
\item litteral
\item värde
\item uttryck
\item identifierare
\item variabel
\item typ
\item tilldelning
\item namn
\item val
\item var
\item def
\item definera och anropa funktion
\item funktionshuvud
\item funktionskropp
\item procedur
\item inbyggda grundtyper
\item Int
\item Long
\item Short
\item Double
\item Float
\item Byte
\item Char
\item String
\item println
\item typen Unit
\item enhetsvärdet ()
\item stränginterpolatorn s
\item if
\item else
\item true
\item false
\item MinValue
\item MaxValue
\item aritmetik
\item slumptal
\item math.random
\item logiska uttryck
\item de Morgans lagar
\item while-sats
\item for-sats\end{itemize}\end{multicols}

\input{modules/w01-intro-exercise.tex}
\input{modules/w01-intro-lab.tex}

\input{modules/w02-programs-chapter.tex}
%\input{generated/w02-chaphead-generated.tex}
\input{modules/w02-programs-exercise.tex}
\input{modules/w02-programs-lab.tex}

\input{modules/w03-functions-chapter.tex}
%\input{generated/w03-chaphead-generated.tex}
\input{modules/w03-functions-exercise.tex}
\input{modules/w03-functions-lab.tex}

\input{modules/w04-objects-chapter.tex}
%\input{generated/w04-chaphead-generated.tex}
\input{modules/w04-objects-exercise.tex}
\input{modules/w04-objects-lab.tex}

\input{modules/w05-classes-chapter.tex}
%\input{generated/w05-chaphead-generated.tex}
\input{modules/w05-classes-exercise.tex}
\input{modules/w05-classes-lab.tex}

\input{modules/w06-patterns-chapter.tex}
%\input{generated/w06-chaphead-generated.tex}
\input{modules/w06-patterns-exercise.tex}
\input{modules/w06-patterns-lab.tex}

\input{modules/w07-sequences-chapter.tex}
%\input{generated/w07-chaphead-generated.tex}
\input{modules/w07-sequences-exercise.tex}
\input{modules/w07-sequences-lab.tex}

\part{Lösningar}

\setcounter{chapter}{11} %L in \Alph
\renewcommand\thechapter{\Alph{chapter}}
\chapter{Lösningar till övningarna}\label{chapter:solutions}

\PreSolutionfalse

\let\QUESTBEGIN\ifPreSolution  % to mark formatting and numbering of exercises
\let\SOLUTION\else  % to mark solutions in the same file as questions
\let\QUESTEND\fi    % to mark end of exercise

\input{modules/w01-intro-exercise.tex}
\input{modules/w02-programs-exercise.tex}
\input{modules/w03-functions-exercise.tex}
\input{modules/w04-objects-exercise.tex}
\input{modules/w05-classes-exercise.tex}
\input{modules/w06-patterns-exercise.tex}
\input{modules/w07-sequences-exercise.tex}

%\chapter{Ordlista}

%\chapter{Lösningar till övningarna}\label{chapter:solutions}
%\foreach \n in {1,...,9}{%
%  \input{modules/w0\n-solutions.tex}
%}
%\foreach \n in {10,...,14}{%
%  \input{modules/w\n-solutions.tex}
%}
%
%\chapter{Snabbreferens}\label{chapter:quickref}
%
%Detta appendix innehåller en snabbreferens för Scala och Java. Snabbreferensen är enda tillåtna hjälpmedel under kursens skriftliga tentamen.
%
%Lär dig vad som finns i snabbreferensen så att du snabbt hittar det du behöver och träna på hur du  effektivt kan dra nytta av den när du skriver program med papper och penna utan datorhjälpmedel.
%
%\clearpage
%~
%\clearpage
%
%\includepdf[pages={1-12}, scale=0.77, frame]{../quickref/quickref.pdf}


\end{document}
