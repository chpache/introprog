%!TEX encoding = UTF-8 Unicode
\documentclass[a4paper, oneside]{compendium}

\usepackage[french]{babel}
\addto\captionsswedish{%
  \renewcommand{\appendixname}{Appendix}%
}

%TODO: Glossary
%http://tex.stackexchange.com/questions/5821/creating-a-standalone-glossary/5837#5837

\setlength{\columnsep}{16mm}

\input{global-constants.tex}

\title{
{\vspace{-3.0cm}\bf\sffamily\fontsize{30}{50}\selectfont  Introduktion till programmering med Scala}
\\ \vspace{1em}%\hspace*{1.5cm}\inputgraphics[width=0.6\textwidth]{../img/gurka} \\
{\sffamily\Huge  Föreläsningar \& uppgifter \\ {\vspace{0.3em}\huge för läsning på skärm}}\\\vspace{1cm}
\includegraphics[height=12cm]{../img/scala-icon.png}
%\includegraphics[height=15cm]{cover/gurka.jpg}
}

\author{\fontsize{14}{17}\selectfont Björn Regnell}
\date{\raggedbottom%
\vspace{1em}\begin{minipage}{1.0\textwidth}\centering\fontsize{14}{17}\selectfont
EDAA45, Lp1-2, HT \CurrentYear\\
Datavetenskap, LTH\\
Lunds universitet\\
~\\
Kompileringsdatum: \today \\
\url{http://cs.lth.se/pgk}
\end{minipage}
}

\usepackage{multicol}

\usepackage{pgffor}  %% http://stackoverflow.com/questions/2561791/iteration-in-latex
                     %  allows:  \foreach \n in {1,...,4}{ do something with \n }

\usepackage{framed}  %  allows:   \begin{framed}\end{framed}
%https://ctan.math.illinois.edu/macros/latex/contrib/framed/framed.pdf

% \newenvironment{Slide}[2][]
%  {\begin{oframed}\setlist{noitemsep}\subsection{#2}}
%  {\end{oframed}}

\newenvironment{Slide}[2][]%
  {\setlist{noitemsep}\subsection{#2}}%
  {~\newline\noindent\rule{\textwidth}{0.4pt}}

%\newcommand{\SlideHeading}[1]{\section*{#1}}
\newcommand{\SlideHeading}[1]{\subsection{#1}}

%\usepackage[most]{tcolorbox}
% \newenvironment{Slide}[2][]
%   {\vspace{0.25em}\begin{tcolorbox}[left=1.5em,%width=1.05\textwidth,
%   grow to right by=0.03\textwidth,grow to left by=0.03\textwidth,%breakable,
%   frame hidden,colback=white]\setlist{noitemsep}\SlideHeading{#2}}
%   {\end{tcolorbox}\vspace{0.25em}}



\newcommand{\Subsection}[1]{} %ignore slide sections
\newcommand{\SlideOnly}[1]{} %ignore slide font size

\newif\ifkompendium  % to allow conditional text in slides only showing up in compendium
\kompendiumtrue      % in slides: \kompendiumfalse


\newif\ifPreSolution  % to allow tasks and solutions in same file
\PreSolutiontrue      % in solutions: \PreSolutionfalse

\let\QUESTBEGIN\ifPreSolution  % to mark formatting and numbering of exercises
\let\SOLUTION\else  % to mark solutions in the same file as questions
\let\QUESTEND\fi    % to mark end of exercise


\input{generated/names-generated.tex}

\makeatletter\@twosidefalse\@mparswitchfalse\makeatother
\geometry{inner=45bp,outer=45bp,top=45bp,bottom=45bp} % better for screen reading
\dottedcontents{section}[4.5em]{}{2.9em}{1pc}
\dottedcontents{subsection}[6.5em]{}{4.2em}{1pc}

%\usepackage{tgheros}
%\renewcommand{\familydefault}{\sfdefault}
\begin{document} 

\fontsize{14pt}{17pt}\selectfont  % bigger front better for screen reading
\pagenumbering{roman}

\frontmatter
\maketitle
\input{prechapters/licence-contributors.tex}
\input{prechapters/progress-forms.tex}
\input{prechapters/preface.tex}

\setcounter{tocdepth}{2} % set headings level in table of contents
\fontsize{13pt}{18pt}\selectfont  % bigger front better for screen reading
\tableofcontents
\mainmatter

\fontsize{13pt}{16pt}\selectfont % bigger front better for screen reading
\pagenumbering{arabic}

\part{Om kursen}
\setcounter{chapter}{-3}
\input{prechapters/course-architecture.tex}
\input{prechapters/course-instructions.tex}
\input{prechapters/how-to-contribute.tex}

%\renewcommand{\SlideHeading}[1]{\subsection{#1}}  %numbering sections in compendium slides

\part{Moduler}
\input{modules/w01-intro-chapter.tex}
\input{modules/w01-intro-exercise.tex}
\input{modules/w01-intro-lab.tex}

\input{modules/w02-programs-chapter.tex}
\input{modules/w02-programs-exercise.tex}
\input{modules/w02-programs-lab.tex}

\input{modules/w03-functions-chapter.tex}
\input{modules/w03-functions-exercise.tex}
\input{modules/w03-functions-lab.tex}

\input{modules/w04-objects-chapter.tex}
\input{modules/w04-objects-exercise.tex}
\input{modules/w04-objects-lab.tex}

\input{modules/w05-classes-chapter.tex}
\input{modules/w05-classes-exercise.tex}
\input{modules/w05-classes-lab.tex}

\input{modules/w06-patterns-chapter.tex}
\input{modules/w06-patterns-exercise.tex}
\input{modules/w06-patterns-lab.tex}

\input{modules/w07-sequences-chapter.tex}
\input{modules/w07-sequences-exercise.tex}
\input{modules/w07-sequences-lab.tex}

\input{modules/w08-matrices-chapter.tex}
\input{modules/w08-matrices-exercise.tex}
\input{modules/w08-matrices-lab.tex}

\input{modules/w09-setmap-chapter.tex}
\input{modules/w09-setmap-exercise.tex}
\input{modules/w09-setmap-lab.tex}

\input{modules/w10-inheritance-chapter.tex}
\input{modules/w10-inheritance-exercise.tex}
\input{modules/w10-inheritance-lab.tex}

\input{modules/w11-scalajava-chapter.tex}
\input{modules/w11-scalajava-exercise.tex}
\input{modules/w11-scalajava-lab.tex}

\input{modules/w12-sorting-chapter.tex}
\input{modules/w12-sorting-exercise.tex}
\input{modules/w12-sorting-lab.tex}

\input{modules/w13-examprep-chapter.tex}
\input{modules/w13-examprep-exercise.tex}
\input{modules/w13-assignment-bank.tex}
\input{modules/w13-assignment-tabular.tex}
\input{modules/w13-assignment-music.tex}
\input{modules/w13-assignment-imageprocessing.tex}

\input{modules/w14-extra-chapter.tex}
\input{modules/w14-extra-exercise.tex}
\input{modules/w14-extra-lab.tex}

\part{Appendix}
\appendix
\input{postchapters/kojo.tex}
\input{postchapters/terminal.tex}
\input{postchapters/compile.tex}
\input{postchapters/debug.tex}
\input{postchapters/document.tex}
\input{postchapters/build.tex}
\input{postchapters/version-control.tex}
\input{postchapters/ide.tex}
\input{postchapters/scalajs.tex} %TODO!!
\input{postchapters/android.tex} %TODO!!
\input{postchapters/vbox.tex}    %TODO!!


\part{Lösningar}

\setcounter{chapter}{11} %L in \Alph
\renewcommand\thechapter{\Alph{chapter}}

\chapter{Lösningar till övningarna}\label{chapter:solutions}

\PreSolutionfalse

\let\QUESTBEGIN\ifPreSolution  % to mark formatting and numbering of exercises
\let\SOLUTION\else  % to mark solutions in the same file as questions
\let\QUESTEND\fi    % to mark end of exercise


\input{modules/w01-intro-exercise.tex}
\input{modules/w02-programs-exercise.tex}
\input{modules/w03-functions-exercise.tex}
\input{modules/w04-objects-exercise.tex}
\input{modules/w05-classes-exercise.tex}
\input{modules/w06-patterns-exercise.tex}
\input{modules/w07-sequences-exercise.tex}

\input{modules/w08-matrices-exercise.tex}
\input{modules/w09-setmap-exercise.tex}
\input{modules/w10-inheritance-exercise.tex}
\input{modules/w11-scalajava-exercise.tex}
\input{modules/w12-sorting-exercise.tex}
\input{modules/w13-examprep-exercise.tex}
\input{modules/w14-extra-exercise.tex}



%\chapter{Ordlista}

%\chapter{Snabbreferens}\label{chapter:quickref}
%
%Detta appendix innehåller en snabbreferens för Scala och Java. Snabbreferensen är enda tillåtna hjälpmedel under kursens skriftliga tentamen.
%
%Lär dig vad som finns i snabbreferensen så att du snabbt hittar det du behöver och träna på hur du  effektivt kan dra nytta av den när du skriver program med papper och penna utan datorhjälpmedel.
%
%\clearpage
%~
%\clearpage
%
%\includepdf[pages={1-12}, scale=0.77, frame]{../quickref/quickref.pdf}

 
\end{document}
